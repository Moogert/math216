\documentclass[12pt, letterpaper]{article}
\usepackage{fullpage}
\usepackage{graphics}

\title{Basic Trigonometry}
\author{DSchafer05\footnote{With much material stolen from Mildorf04}}

\begin{document}

\maketitle

\section{Fundementals}

\subsection{Trigonometric Functions}

\par There are three basic trigonometric functions, sine, cosine and tangent, whose definitions can be easily observed in a right triangle.  Where $o=$ the opposite side, $a=$ the adjacent side and $h=$ the hypotenuse, 

\[
\sin{\theta} = \frac{o}{h}
\]
\[
\cos{\theta} = \frac{a}{h}
\]
\[
\tan{\theta} = \frac{o}{a} = \frac{sin(\theta)}{cos(\theta)}
\]

\par This is easily remembered using the acronynm SOHCAHTOA.

\par The other 3 trigonometric functions, cosecant, secant and cotangent, are defined in terms of the first three.

\[
\csc{\theta} = \frac{1}{\sin{\theta}}
\]
\[
\sec{\theta} = \frac{1}{\cos{\theta}}
\]
\[
\cot{\theta} = \frac{\csc{\theta}}{\sec{\theta}} = \frac{\cos{\theta}}{\sin{\theta}} = \frac{1}{\tan{\theta}}
\]

\subsection{Trigonometric Values}

\par The following values, derived by analyzing 30-60-90 and 45-45-90 triangles, are the basis for most of the trigonometry encountered in contest math.  It is quite important that these values be memorized completely.

\begin{tabular}{lll}
$\sin{0} = 0$ & $\cos{0} = 1$ & $\tan{0} = 0$ \\
$\sin{\frac{\pi}{6}} = \frac{1}{2}$ & $\cos{\frac{\pi}{6}} = \frac{\sqrt{3}}{2}$ & $\tan{\frac{\pi}{6}} = \frac{1}{\sqrt{3}}$ \\
$\sin{\frac{\pi}{4}} = \frac{\sqrt{2}}{2}$ & $\cos{\frac{\pi}{4}} = \frac{\sqrt{2}}{2}$ & $\tan{\frac{\pi}{4}} = 1$ \\
$\sin{\frac{\pi}{3}} = \frac{\sqrt{3}}{2}$ & $\cos{\frac{\pi}{3}} = \frac{1}{2}$ & $\tan{\frac{\pi}{3}} = \sqrt{3}$ \\
$\sin{\frac{\pi}{2}} = 1$ & $\cos{\frac{\pi}{2}} = 0$ & $\tan{\frac{\pi}{2}} = \infty$ \\
\end{tabular}

\subsection{Simple Trigonometric Identities}

\par The basic identities given here allow any trigonometric functions to be evaluated, as long as $\theta | \frac{\pi}{6}$.

\begin{tabular}{ll}
$\sin{(\frac{\pi}{2}-\theta)} = \cos{\theta}$ & $\cos{(\frac{\pi}{2}-\theta)} = \sin{\theta}$ \\
$\sin{(-\theta)} = -\sin{\theta}$ & $\cos{(-\theta)} = \cos{\theta}$ \\
$\sin{(\pi-\theta)} = \sin{\theta}$ & $\cos{(\pi-\theta)} = -\cos{\theta}$ \\
$\sin^2{\theta} + \cos^2{\theta} = 1$ & \\
\end{tabular}

\section{Advanced Computational Concepts}

\subsection{Addition and Subtraction Formulae}

\par To derive the values of the trigonometric functions for different $\theta$, it is neccessary to use addition and subtraction formulae.  These formulae are extremely important, and are well worth memorizing.

\[
\sin{(\theta + \phi)} = \sin{\theta}\cos{\phi} + \cos{\theta}\sin{\phi}
\]
\[
\cos{(\theta + \phi)} = \cos{\theta}\cos{\phi} - \sin{\theta}\sin{\phi}
\]
\[
\tan{(\theta + \phi)} = \frac{\tan(\theta) + \tan(\phi)}{1-\tan{\theta}\tan{\phi}}
\]

\par Using the Simple Trigonometric Identities above, Subtraction Formulae can also be derived.  These are listed here for reference.

\[
\sin{(\theta - \phi)} = \sin{\theta}\cos{\phi} - \cos{\theta}\sin{\phi}
\]
\[
\cos{(\theta - \phi)} = \cos{\theta}\cos{\phi} + \sin{\theta}\sin{\phi}
\]
\[
\tan{(\theta - \phi)} = \frac{\tan(\theta) - \tan(\phi)}{1+\tan{\theta}\tan{\phi}}
\]

\subsection{Double-Angle and Half-Angle Formulae}

\par The Double-Angle formulae are created directly from the Addition Formulae above, setting $\theta = \phi$.

\[
\sin{(2\theta)} = 2\sin{\theta}\cos{\theta}
\]
\[
\cos{(2\theta)} = \cos^2{\theta} - \sin^2{\theta}
\]
\[
\tan{(2\theta)} = \frac{2\tan{\theta}}{1-\tan^2{\theta}}
\]

\par These formulae are used to derive the half-angle formulae.

\[
\sin{\frac{\theta}{2}} = \pm\sqrt{\frac{1-\cos^2{(2\theta)}}{2}}
\]
\[
\cos{\frac{\theta}{2}} = \pm\sqrt{\frac{1+\cos^2{(2\theta)}}{2}}
\]
\[
\tan{\frac{\theta}{2}} = \pm\frac{\sin{\theta}}{1+\cos{\theta}}
\]

\subsection{Sum-Product Formulae}

\par These formulae are usually not needed for trigonometry problems, but they can often make life a lot easier.

\[
\sin(\alpha)\sin(\beta) = \frac{\cos(\alpha - \beta) - \cos(\alpha
  + \beta)}{2}
\]
\[
\cos(\alpha)\cos(\beta) = \frac{\cos(\alpha - \beta) + \cos(\alpha
  + \beta)}{2}
\]
\[
\sin(\alpha)\cos(\beta) = \frac{\sin(\beta + \alpha) - \sin(\beta
  - \alpha)}{2}
\]

\section{Trigonometric Theory}

\subsection{Trigonometric Laws}

\par In the statement of these laws, $a$, $b$, and $c$ are the sides of a triangle, and $A$, $B$, and $C$ are the angles opposite those sides.

\[
\frac{a}{\sin{A}} = \frac{b}{\sin{B}} = \frac{c}{\sin{B}} = 2R
\]
\[
c^2 = a^2 + b^2 - 2ab\cos{C}
\]

\subsection{Identities}

\par This section containse some identities that often are useful with contest math.

\par In a triangle $ABC$,

\[
\tan{A}\tan{B}\tan{C} = \tan{A} + \tan{B} + \tan{C}
\]
\[
\cos^2{A}+\cos^2{B}+\cos^2{C}+2\cos{A}\cos{B}\cos{C} = 1
\]

\subsection{Rocco's Method}

\par Rocco Repeski (TJ Class of 2004) developed this method for solving trig problems.  This should only be used as an absolute last resort, given that it doesn't usually work; but if time is running out and no other solution method appears, it's better than nothing.

\par The theory behind Rocco's method is that any trig functions can be changed to a known trig function.  For example, what is $\cos{50^{o}}$?  You probably have no idea!  Rocco's method says that $\cos{50^{o}}$ = $\cos{45^{o}}^-$, because the the cosine will get slightly smaller going from $45^{o}$ to $50^{o}$.  Hopefully, by the end of the problem, the +s and -s will cancel out, and a nice, pretty answer will result.  This rarely works, but at the worst provides a reasonably close guess.

\par For an example of a problem solvable using Rocco's Method, see the final question in the problem set.

\section{Problems}

\begin{enumerate}

\item (Traditional) $\cos{x} + \sin{x} = .5$.  Solve for $\sin{2x}$.

\item (Schafer06) Compute $\displaystyle\sum_{i=1}^{90}{\sin{i^{o}}} + \sum_{i=91}^{180}{\cos{i^{o}}}$.

\item (Traditional) $\tan{20^{o}}\tan{40^{o}}\tan{80^{o}} = \tan{x^{o}}$.  Solve for $x$.

\end{enumerate}

\newpage

\section{Hints}

All of the these hints are written in order to avoid giving away the solution, but it is still better to try each problem for a reasonable amount of time before using these hints.

\begin{enumerate}

\item Consider simple trigonometric identities.  What can be done to the given to tranform it to an identity?

\item Clearly, adding all of the numbers up isn't possible.  What can be done to simplify the problem?

\item Seperate out $\tan{\theta}$ into $\frac{\sin{\theta}}{\cos{\theta}}$.  If that fails, try Rocco's Method (this is one of the few problems where it works!)

\end{enumerate}

\newpage

\section{Solutions}

It is highly recommended that you not look at the solutions until you have put forth a strong effort to solve the problems yourself.  

\begin{enumerate}

\item \textbf{$-.75$}.  Squaring both sides, $(\cos{x} + \sin{x})^2 = \cos^2{x} + 2\cos{x}\sin{x} + \sin^2{x} = .25$.  As $2\sin{x}\cos{x} = \sin{2x}$, $1 + 2\cos{x}\sin{x} = .25$, and $\sin{2x} = -.75$.

\item \textbf{$0$}.  Remember that $\cos(90^{o}+\theta) = \sin{-\theta} = -\sin{\theta}$.  Thus, the expression given is equivelent to $\displaystyle\sum_{i=1}^{90}{\sin{i^{o}}} + \sum_{i=1}^{90}{-\sin{i^{o}}} = 0$.

\item \textbf{$60$}.

\par \textbf{Method 1: Actual Trig}. Expand the expression: $\tan{20^{o}}\tan{40^{o}}\tan{80^{o}} = \frac{\sin{20^{o}}\sin{40^{o}}\sin{80^{o}}}{\cos{20^{o}}\cos{40^{o}}\cos{80^{o}}}$.  Using the Product-Sum formulae, this expression is shown to equal $\frac{(\cos{20^{o}}-\cos{60^{o}})\sin{80^{o}}}{(\cos{20^{o}} + \cos{60^{o}})\cos{80^{o}}}$.  Multiplying out again and using the Product-Sum formulae, this equals $\frac{\cos{20^{o}}\sin{80^{o}} - \cos{60^{o}}\sin{80^{o}}}{\cos{20^{o}}\cos{80^{o}} + \cos{60^{o}}\cos{80^{o}}} = \frac{(\frac{\sin{100^{o}}-\sin{(-60^{o})}}{2}) - (\frac{\sin{80^{o}}}{2})}{(\frac{\cos{100^{o}}+\cos{60^{o}}}{2}) + (\frac{\cos{80^{o}}}{2})}$, and as $\cos{100^{o}} = -\cos{80^{o}}$ and $\sin{100^{o}} = \sin{80^{o}}$, $\frac{(\frac{\sin{60^{o}}}{2}) + (\frac{\sin{100^{o}}-\sin{80^{o}}}{2})}{(\frac{\cos{60^{o}}}{2}) + (\frac{\cos{100^{o}}+\cos{80^{o}}}{2})} = \frac{\sin{60^{o}}}{\cos{60^{o}}} = \tan{60^{o}}$.

\par \textbf{Method 2: Rocco's Method}.  Simplifying, $\tan{20^{o}}\tan{40^{o}}\tan{80^{o}} = \tan{30^{o}}^{-}\tan{60^{o}}^{-}\tan{60^{o}}^{++} = \tan{60^{o}}$.  This really shouldn't work, but it does.

\end{enumerate}

\end{document}
