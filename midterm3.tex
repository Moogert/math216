\documentclass[10pt,letterpaper]{article}
\usepackage[letterpaper,margin=0.5cm]{geometry}
\usepackage[utf8]{inputenc}
\usepackage{amsmath}
\usepackage{amsfonts}
\usepackage{pdfpages}
\usepackage{amssymb}
\usepackage{siunitx}
\author{Jeffrey Wubbenhorst}
\title{Math 216 Midterm 3 Study Guide }

\begin{document}
\maketitle

\section{Change of Basis}

\begin{itemize}
\item For some basis $v$ and the standard basis $s$, the matrix $[I]^s_v$ is the change of basis matrix $s \to v
$

\end{itemize}

\section{Matrices for Linear Transformations}
% as in section 5.3 in the book 
\begin{itemize} 
\item Suppose $T: V\to W$ is an LT; further suppose $v_1, ...,v_n$ form a basis $\alpha$ for $V$ and $w_1, ..., w_m$ form a basis $\beta$ for $W$. If we were to express $T(v_1),...,T(v_n)$ in terms of 
$w_1,...,w_m$, we get a series of equations of the form: 
$$
T(v_1) =a_{11}w_1+a_21w_2+...+a_{m1}w_{m} \\
T(v_2) = a_{12}w_1+a_{22}w_2+...+a_{m2}w_m \\
\vdots \\
T(v_m)=a_{1n}w_1
$$

\end{itemize}

\section*{Chapter 9.3 (``Schur's Theorem and Symmetric Matrices")}

\begin{itemize}

\item Recall tha $B$ is similar to $A$ if there exists an invertible $n\times n$ matrix such that $$B=P^{-1}AP$$
If $P$ is an orthogonal matrix, that is: 
$$P^{-1}=P^T \mbox{ and } B=P^TAP$$ 
...we say that $B$ is \textbf{orthogonally simliar} to $A$. 

\item If $P$ is an orthogonal matrix, $B$ is an orthogonal basis for $\mathbb{R}^n. $ \footnote{Since $P$ is change of basis matrix of $T(x)=Ax$}

\item \textbf{Schur's Theorem:} Suppose $A$ is an $n\times n$ matrix. If all the eigenvalues of $A$ are real numbers, $A$ is orthogonally similar to an upper triangle matrix. 

\item If $C$ is a matrix whose entries are complex numbers, the \textbf{Hermitian conjugate} of $C$, notated as $C^*$, is given as $C^*=\bar{C}^T$

\item An $n \times n$ matrix $P$ with complex entries is a \textbf{unitary matrix }if $P^*P=I$. Unitary matrices are the analog of orthogonal matrices in a complex space. 

\item An $n\times n$ matrix $B$ is unitaritly similar to an $n\times n$ matrix $A$ if there exists a unitary $P$ such that $B=P^TAP$. 

% check these... are all conditions met?
\item If $A$ is $n \times n$ and symmetric with real entries, all eigenvalues of $A$ are real. 

\item If $A$ is symmetric with real entries, $A$ is diagonalizable. 

\item If $A$ is symmetric with real entries and $\vec{v_1},\vec{v_2}$ are eigenvectors of $A$ with different associated eigenvalues, $v_1$ is orthogonal to $v_2$. 

\item Commonly used steps for finding an orthogonal matrix $P$ that diagonalizes an $n\times n$ matrix symmetrix matrix $A$ with real entries: 

\begin{enumerate}
\item Find bases for eigenspaces of $A$
\item Apply Gram-Schmidt process to basis of each eigenspace to obtain an orthonormal basis. 
\item $P$ is a matrix made of columns from step 2. \textit{YEET }
\end{enumerate}

\item If $A$ Is an $n \times n$ symmetric matrix with real entries, then all the eigenvalues of a $A$ are real, and all eigenspaces have real bases. 
% starting from bray notes here. Page 254. 

\item In the \textit{complex space }$\mathbb{C}$, we have to redefine the \textbf{inner product}. The \textbf{Hermitian dot product} on $\mathbb{C}$ is defined as: 
$$<\vec{v},\vec{w}>=\sum v_i\bar{w_i}=\vec{v}^T\vec{w}$$
The \textbf{properties of the Hermitian dot product }are: 
\begin{enumerate}
\item $<\vec{v}, \vec{w}>_H=<\vec{w}, \vec{v}>_H$
\item $<\vec{u}+\vec{v},\vec{w}>_H=<\vec{u},\vec{w}>+<\vec{v},\vec{v},\vec{w}>_H$
\item $<c\vec{v},\vec{w}>_H=c<\vec{v},\vec{w}>_H$
\item $<\vec{v},\vec{v}>_H\geq , <\vec{v},\vec{v}>_H=0 \mbox{ iff } \vec{v}=0$
\end{enumerate}
Functions satisfying these properties are called \textbf{Hermitian inner products.}

\item With inner product spaces, we define the angle between two vectors as: 
$$\theta = \cos^{-1}\frac{u \cdot v}{||u||||v||}$$

\item One common (and important!) inner product is the $L^2$ inner product, which is defined as: 
$$<f,g> = \int f\cdot g dx$$

\item Magnitude with inner products is defined as 
$$||v|| = \sqrt{<v,v>}$$

\item With the Hermitian inner product, we also have the \textbf{Hermitian transpose:}
$$A^*=\bar{A}^T$$ 
Note that $\mbox{Hermitian transpose } \leftrightarrow \mbox{ Hermitian conjugate } \leftrightarrow \mbox{ adjoint }$. 
\item Transposes relate to symmetry by definition: 
$$ <A\vec{v},\vec{w}>_H=<\vec{w},A\vec{v}>_H$$ 
Real symmetric matrices are Hermitian as well. 

\item if $AA$ is real and symmetric with eigenvalues $\lambda_1 \neq \lambda_2$ and associated eigenvectors $\vec{v_1}, \vec{v_2}$, $\vec{v_1}, \vec{v_2}$ are \textbf{orthogonal}. 

\item $<A\vec{v}, \vec{w}>_H=<\vec{v},A^*\vec{w}>_H$

\item Recall that $A, B$ are similar if $B=P^{-1}AP$, where $P$ is a change of basis. We now say that: 
If $A,B$ are similar by $BP^{-1}AP$ and $P$ is orthogonal, then $A, B$ are orthogonally similar, and $B=P^TAP$. 
\item If $A$ is diagonalizable with $D=P^{-1}AP$ (that is, columns of $P$ are a basis of eigenvector) and if $P$ is orthogonal, then we say $A$ is \textbf{orthogonally diagonalizable}. 

\item Every real, symmetric matrix is \textbf{orthogonally diagonalizable}. 
\end{itemize}

\section*{}

\section*{Chapter 6.1 (Theory of Systems of LDEs)}

\begin{itemize}
\item If $a_{i,j}(x) \mbox{ and } g_i(x) \mbox{ are continuous on }(a,b)$ containing $x_0$, for $a\leq i \leq n, 1\leq j \leq n$, the IVP 
$$Y'=A(x)Y+G(x), Y(x_0)=\left[
\begin{array}{c}
b_1 \\ 
\vdots  \\ 
b_n \\ 
\end{array}
\right]
 $$
 
has a unique solution on $(a,b)$ 

\item The solutions to a homogenous system of first-order LDEs 
$$Y' = A(x)Y$$ form a vector space (subspace) of dimension $n$ - the basis for the subspace is a fundamental set of solutions 

\item A set of $n$ linearly independent solutions $Y_1, Y_2, ..., Y_n$ to a homogenous system of $n$ first-order LDEs is a fundamental set of solutions, often notated as $M$. % redundant? 

\item The general solution to a homogenous system of 1st-order linear differential equations is written as 
$$Y_H=MC$$ 

\item If $Y_1, Y_2, ..., Y_n$ form a fundamental set of solutions to a system of homogenous first-order linear differential equations 
$Y'=A(x)Y$
...then ever solution to this nonhomogenous system is of the form: 
$$Y = Y_H + Y_P = c_1Y_1 +...+ c_nY_n+Y_p=MC+Y_p$$

\item If the \textbf{Wronskian} is inequal to 0, functions are linearly independent. If the Wronskian is 0 for some $x_0\in (a,b)$, then $Y_1, ... , Y_n$ is \textbf{linearly dependent}. \\ 
This further implies that, for the fundamental set of solutions over $(a,b)$, $w(Y_1(x), Y_2(x),...,Y_n(x))\neq 0 \forall x\in (a,b)$
\end{itemize}

\section*{Chapter 6.2 (Constant Coefficient Homogenous Systems  (Diagonalizable)}

\begin{itemize}

\item A change of basis can simplify a system as well as a matrix. Diagonal systems are also easy to solve, and we like easy around here. \footnote{Hence Duke math.} 

\item Diagonal systems are \textbf{decoupled}, which means that the variables we're solving for don't interact with each other, and can be solved for independently. 


\item If $A,B$ are similar matrices with $B=P^{-1}AP$ where $P$ is an invertible $n\times n$ matrix. If $Z$ is a solution of $Y'=BY$, then $PZ$ is a solution of $Y'=AY$. \\
Also, if $Z_1, Z_2, ..., Z_n$ forms a fundamental solution to $Y'=BY$, then $PZ_1,PZ_2,...,PZ_n$ forms a fundamental set of solutions to $Y'=AY$

\item To rephrase this: if $M_B= [Z_1,  Z_2, ..., Z_n]$ is a matrix of fundamental solutions for $Y'=BY$, then 
$$M_A = PM_B = [PZ_1, PZ_2,...,PZ_n]$$ 
is a matreix of fundamental solutions to $Y'=AY$. 

\item Suppose a matrix $A$ is diagonalizable and that there's some $P$ such that $D=P^{-1}AP$ is the diagonal matrix: 
$$D = \left[
\begin{array}{ccccc}
d_1 & 0 & 0 & \dots & 0  \\ 
0 & d_2 & 0 & \dots & 0 \\ 
\vdots & \vdots & \vdots & & \vdots \\ 
0 & 0 & 0 & \dots & d_n \\
\end{array}
\right]$$

...which imples that the general solution to $Y'=AY$ is: 

$$ P\left[
\begin{array}{c}
c_1e^{d_1x} \\
c_2e^{d_2x} \\ 
\vdots \\ 
c_ne^{d_nx}
\end{array}
\right]
$$
\textit{Note: you'll really, really want to be familiar with this. In the book, look at page 303.}

\item If $U(x) + iV(x)$ is a solution to $Y'=A(x)Y$, then $U(x) \mbox{ and } V(x)$ are solutions to $Y'=A(x)Y$. 

\item If some $A$ is diagonalizable with 
$$D = P^{-1}AP, A=PDP^{-1}$$ 
...we can rewrite $$\vec{y}'=A\vec{y}$$ as 
$$\vec{y}'=PDP^{-1}$$
and multiply to obtain a diagonal system $\vec{y}' = PDP^{-1}\vec{y}$
...which eventualy becomes a decoupled system $\vec{Z}'=D\vec{Z}$, which,
after many more steps, becomes a fundamental set of solutions: 
$\{ e^{\lambda_1x}\vec{e1},...e^{\lambda _nx}\vec{e_n}\}$

\item Consider $\vec{y}'=A\vec{v}$, some diagonalizable $A$ with eigenvalues $\lambda_1,\lambda_2,...\lambda_n$ and eigenvectors $\vec{v_1},\vec{v_2},...,\vec{v_n}$. Then 
$$\{e^{\lambda_1x}v_1,...,e^{\lambda_nx}\vec{v_n}\}$$
forms a fundamental set of solutions. 

\end{itemize}

\section*{Diagonal matrices}

\begin{itemize}
\item A matrix can be \textbf{\textit{diagonalized}} iff there exists a basis $\beta$ for $\mathbb{R}^n$ made of eigenvectors. The entries of the diagonal matrix are the eigenvalues. 

\item For a diagonal matrix $D$: 
$$D = P^{-1}AP$$


\end{itemize}


\end{document}

